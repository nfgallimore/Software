\documentclass[11pt]{article}

\usepackage{amsfonts,amssymb,fullpage,enumerate}

\newcommand{\psnum}{2}
\newcommand{\assdate}{Sep. 13, 2013 \hspace{-5em}}
\newcommand{\duedate}{Sep. 20, 2013}
\def\inclsolns{0}

\newtheorem{theorem}{Theorem}
\newtheorem{conjecture}[theorem]{Conjecture}
\newtheorem{definition}[theorem]{Definition}
\newtheorem{lemma}[theorem]{Lemma}
\newtheorem{proposition}[theorem]{Proposition}
\newtheorem{corollary}[theorem]{Corollary}
\newtheorem{claim}[theorem]{Claim}
\newtheorem{fact}[theorem]{Fact}
\newtheorem{openprob}[theorem]{Open Problem}
\newtheorem{remk}[theorem]{Remark}
\newtheorem{apdxlemma}{Lemma}

\newenvironment{remark}{\begin{remk}

\begin{normalfont}}{\end{normalfont}
\end{remk}}
\newtheorem{sublemma}[theorem]{Sublemma}


%%%%%%%%%%%%%%%%%%%% proof environments

\def\FullBox{\hbox{\vrule width 8pt height 8pt depth 0pt}}

\def\qed{\ifmmode\qquad\FullBox\else{\unskip\nobreak\hfil
\penalty50\hskip1em\null\nobreak\hfil\FullBox
\parfillskip=0pt\finalhyphendemerits=0\endgraf}\fi}

\def\qedsketch{\ifmmode\Box\else{\unskip\nobreak\hfil
\penalty50\hskip1em\null\nobreak\hfil$\Box$
\parfillskip=0pt\finalhyphendemerits=0\endgraf}\fi}

\newenvironment{proof}{\begin{trivlist} \item {\bf Proof:~~}}
  {\qed\end{trivlist}}

\newenvironment{proofsketch}{\begin{trivlist} \item {\bf
Proof Sketch:~~}}
  {\qedsketch\end{trivlist}}

\newenvironment{proofof}[1]{\begin{trivlist} \item {\bf Proof
#1:~~}}
  {\qed\end{trivlist}}

\newenvironment{claimproof}{\begin{quotation} \noindent
{\bf Proof of claim:~~}}{\qedsketch\end{quotation}}


%%%%%%%%%%%%%%%%%%%%%%% text macros
\newcommand{\etal}{{\it et~al.\ }}
\newcommand{\ie} {{\it i.e.,\ }}
\newcommand{\eg} {{\it e.g.,\ }}
\newcommand{\cf}{{\it cf.,\ }}

%%%%%%%%%%%%%%%%%%%%%%% general useful macros
\newcommand{\eqdef}{\mathbin{\stackrel{\rm def}{=}}}
\newcommand{\R}{{\mathbb R}}
\newcommand{\N}{{\mathbb{N}}}
\newcommand{\Z}{{\mathbb Z}}
\newcommand{\poly}{{\mathrm{poly}}}
\newcommand{\loglog}{{\mathop{\mathrm{loglog}}}}
\newcommand{\zo}{\{0,1\}}
\newcommand{\suchthat}{{\;\; : \;\;}}
\newcommand{\pr}[1]{\Pr\left[#1\right]}
\newcommand{\deffont}{\em}
\newcommand{\getsr}{\mathbin{\stackrel{\mbox{\tiny R}}{\gets}}}
\newcommand{\E}{\mathop{\mathrm E}\displaylimits}
\newcommand{\Var}{\mathop{\mathrm Var}\displaylimits}
\newcommand{\eps}{\varepsilon}


%%%%%%%%%%%%%%%%%%% macros particular to this course
% for author notes
\newcommand{\authnote}[2]{{ \bf [#1's Note: #2]}}
\newcommand{\Snote}[1]{{\authnote{Salil}{#1}}}
\newcommand{\Mnote}[1]{{\authnote{Minh}{#1}}}

\def\textprob#1{\textmd{\textsc{#1}}}
\newcommand{\mathprob}[1]{\mbox{\textmd{\textsc{#1}}}}
\newcommand{\SAT}{\mathprob{SAT}}
\newcommand{\yes}{{\sc yes}}
\newcommand{\no}{{\sc no}}
\newcommand{\QuadRes}{\textprob{Quadratic Residuosity}}
\newcommand{\QuadNonres}{\textprob{Quadratic Nonresiduosity}}

\newcommand{\class}[1]{\mathbf{#1}}
\newcommand{\SZK}{\class{SZK}}
\newcommand{\BPP}{\class{BPP}}
\newcommand{\NP}{\class{NP}}
\newcommand{\IP}{\class{IP}}
\renewcommand{\P}{\class{P}}
\newcommand{\negl}{{\mathrm{neg}}}

\newcommand{\Enc}{\mathsf{Enc}}
\newcommand{\Dec}{\mathsf{Dec}}
\newcommand{\Gen}{\mathsf{Gen}}
\newcommand{\Tag}{M}
\newcommand{\Sign}{\mathrm{S}}
\newcommand{\Ver}{V}
\newcommand{\Commit}{\mathrm{Com}}
\newcommand{\Com}{\mathrm{Com}}
\newcommand{\tagsymbol}{t}


\newcommand{\MsgSp}{\mathcal{M}}
\newcommand{\KeySp}{\mathcal{K}}
\newcommand{\CiphSp}{\mathcal{C}}
\newcommand{\calA}{\mathcal{A}}

\newcommand{\key}{k}
\newcommand{\td}{t}

\newcommand{\DIV}{\mathrm{DIV}}
\newcommand{\EXP}{\mathrm{EXP}}
\newcommand{\MODEXP}{\mathrm{MODEXP}}
\newcommand{\GCD}{\mathrm{GCD}}


\newcommand{\Dist}{\mathcal{D}}
\newcommand{\LR}{\mathrm{LR}}
\newcommand{\Oracle}{\mathrm{Oracle}}
\newcommand{\Adv}{\mathrm{Adv}}
\newcommand{\DES}{\mathrm{DES}}
\newcommand{\AES}{\mathrm{AES}}
\newcommand{\FFam}{\mathcal{F}}
\newcommand{\HFam}{\mathcal{H}}
\newcommand{\Rand}{\mathcal{R}}
\newcommand{\Ind}{\mathcal{I}}
\newcommand{\Dom}{D}
\newcommand{\Rng}{R}
\newcommand{\DLog}{\mathrm{DLog}}
\newcommand{\QR}{\mathrm{QR}}
\newcommand{\QNR}{\mathrm{QNR}}
\newcommand{\half}{\mathrm{half}}
\newcommand{\lsb}{\mathrm{lsb}}
\newcommand{\IV}{\mathrm{IV}}
\newcommand{\Field}{\mathbb{F}}
\newcommand{\PK}{\mathit{PK}}
\newcommand{\SK}{\mathit{SK}}
\newcommand{\pk}{\mathit{pk}}
\newcommand{\sk}{\mathit{sk}}
% \newcommand{\key}{\mathsf{key}}

\newcommand{\accept}{\mathtt{accept}}
\newcommand{\reject}{\mathtt{reject}}
\newcommand{\fail}{\mathtt{fail}}
\newcommand{\MD}[1]{\mathrm{MD{#1}}}
\newcommand{\SHA}{\mbox{SHA-1}}

\newcommand{\pf}{\mathit{proof}}
\newcommand{\compind}{\mathbin{\stackrel{\rm
c}{\equiv}}}

\newcommand{\Ideal}{\mathbf{Ideal}}
\newcommand{\Real}{\mathbf{Real}}
\newcommand{\mvec}{\overline{m}}

\newcommand{\View}{\mathsf{View}}
\newcommand{\ThreeCol}{\textprob{Graph 3-Coloring}}
\newcommand{\TCOL}{\mathprob{3COL}}

\newcommand{\OT}{\mathrm{OT}}


\newcounter{problem}
\newenvironment{problem}[1]{\stepcounter{problem}
\paragraph{Problem \theproblem. #1}}{}

\ifnum\inclsolns=1
\newenvironment{solution}{\paragraph{Solution.}}{}
\else
\newenvironment{solution}{\begin{remove}}{\end{remove}}
\fi

\pagestyle{plain}

%------------------------------------------------------------------------------$
\begin{document}

\begin{center}
\renewcommand{\arraystretch}{2}
\begin{tabular}{|c|}
\hline
{\large \bfseries CS 127/CSCI E-127: Introduction to Cryptography} \\

{\large \bfseries Problem Set \psnum}\\
Assigned: \assdate
\hspace{20em} Due: \duedate\ (5:00 PM)\\
\hline
\end{tabular}
\renewcommand{\arraystretch}{1}
\end{center}
\vspace{1cm}

\noindent Justify all of your answers.  See the syllabus for
collaboration and lateness policies. Submit solutions by email to {\tt
mbun@seas} (and please put the string ``CS127PS2'' somewhere in your subject line).

\begin{problem}{(More examples of perfect secrecy)}
In this question, suppose we wish to encrypt messages over the English alphabet $\Sigma = \{a, \dots, z\}$, which we can view as $\{0, 1, \dots, 25\}$.
\begin{enumerate}[a.]
\item Prove that the shift cipher for messages of length 1 over $\Sigma$
satisfies the definition of perfect secrecy.
\item What is the largest message space $\MsgSp \subseteq \Sigma^*$ for which the mono-alphabetic substitution cipher provides perfect secrecy? Justify your answer (proving both perfect secrecy and that no larger message space is possible).
\item Prove that the Vigen\`{e}re cipher using fixed period $t$ is perfectly secret when used to encrypt messages of length $t$.
\end{enumerate}
\end{problem}


\begin{problem}{(Statistical security)}
Recall that  $(\Gen,\Enc,\Dec)$ has {\em statistically
$\eps$-indistinguishable encryptions} if for every two
$m_0,m_1\in\MsgSp$ and every $T\subseteq \CiphSp$,
$$\left|\pr{\Enc_K(m_0)\in T} - \pr{\Enc_K(m_1)\in T}\right| \leq
\eps,$$ where the probabilities are taken over $K\getsr \Gen$ and
the coin tosses of $\Enc$.

\begin{enumerate}[a.]
\item Show that statistical $0$-indistinguishability is equivalent
to perfect indistinguishability.

\item In analogy with adversarial indistinguishability, we say that an encryption scheme $(\Gen, \Enc, \Dec)$ satisfies \emph{$\eps$-adversarial indistinguishability} if every adversary $\calA$ succeeds at the eavesdropping indistinguishability game defined in class with probability at most $\frac{1 + \eps}{2}$:
\begin{enumerate}[1.]
\item $\calA$ outputs a pair of messages $m_0, m_1 \in \MsgSp$.
\item A random key $K \getsr \Gen$ and a bit $b \getsr \{0, 1\}$ are sampled. The ciphertext $c \getsr \Enc_K(m_b)$ is computed and given to $\calA$.
\item $\cal A$ outputs a bit $b'$ and succeeds iff $b = b'$.
\end{enumerate}
Show that if the encryption scheme $(\Gen, \Enc, \Dec)$ has statistically $\eps$-indistinguishable encryptions, then it also satisfies $\eps$-adversarial indistinguishability. (The converse is also true, as we will discuss in section.)
\end{enumerate}
For the next three parts, suppose $(\Gen,\Enc,\Dec)$  has
statistically $\eps$-indistinguishable encryptions for message space
$\MsgSp$. Below you will prove that the number of keys must be at
least $(1-\eps)\cdot |\MsgSp|$, so statistical security doesn't help
much to overcome the limitations of perfect secrecy.

\begin{enumerate}[a.]
\stepcounter{enumi}
\item Call a ciphertext $c$ {\em decryptable} to $m\in \MsgSp$ if
there is a key $K$ such that $\Dec_K(c)=m$.  Prove that for every
two messages $m,m'\in\MsgSp$,
$$\pr{\mbox{$\Enc_K(m)$ is decryptable to $m'$}}\geq 1-\eps,$$
where the probability is taken over $K\getsr \Gen$ and the coin
tosses of $\Enc$.

 \item Show that for every message $m\in \MsgSp$,
$$\E\left[\#\{ m' : \mbox{$\Enc_K(m)$ is decryptable to
$m'$}\right] \geq (1-\eps)\cdot |\MsgSp|,$$ where again the
probability is taken over $K$ and the coin tosses of $\Enc$. (Hint:
for each $m'$, define a random variable $X_{m'}$ that equals $1$ if
$\Enc_K(m)$ is decryptable to $m'$, and equals $0$ otherwise.)

\item Conclude that the number of keys must be at least
$(1-\eps)\cdot |\MsgSp|$.
\end{enumerate}
\end{problem}

\begin{problem}{(Factorization is ``$\NP$-easy'')}
\begin{enumerate}[a.]
\item Let $L= \{ (x,y) \in \N \times \N:  \textrm{$x$ has a factor between
$2$ and $y$}\}$. Show that the language $L$ is in $\NP$.
\item Show that if $L$ is in $\P$, then
there is a polynomial-time algorithm for the integer factorization problem: given a composite number $x$, find a nontrivial factor of $x$.
Thus, if $\P=\NP$, then factorization is easy.
\end{enumerate}
\end{problem}

\begin{problem}{(Reducing the error of randomized algorithms)}
Suppose we have randomized algorithm for computing a function $f$
which gives an incorrect answer with probability $\leq 1/3$, and we want
to reduce its error by repeating it several times and taking a majority vote.
Use the Chernoff Bound to estimate how many repetitions suffice to reduce the error probability to $1/1000$.  And to $2^{-k}$?
\end{problem}

\end{document}
